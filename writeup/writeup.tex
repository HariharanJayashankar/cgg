\documentclass[a4paper 12pt]{article}
\usepackage[hmargin=1.2in,vmargin=1in]{geometry}
%\usepackage[latin1]{inputenc}
\usepackage{enumerate}
\usepackage{multirow}
\usepackage{url}
\usepackage{array}
\usepackage{bbm}
\usepackage{setspace}
\usepackage{color}
\usepackage[colorlinks=false,urlcolor=blue]{hyperref}
\usepackage{amsfonts} % Worcester's
\usepackage{appendix} %Added Jan 2012
\usepackage{graphicx,caption}
\usepackage{subcaption}
\usepackage[authoryear,round]{natbib}
\usepackage{endnotes}
\usepackage{amsmath, amssymb,epstopdf,sgame,booktabs}
\usepackage{ntheorem}
%\usepackage{amssymb,mathrsfs,multirow,sgame,lscape,subfig,enumitem,float,array,booktabs}
\usepackage{amssymb,mathrsfs,multirow,sgame,pdflscape,float,array,booktabs}
\usepackage{listings}
\usepackage{color} %red, green, blue, yellow, cyan, magenta, black, white
\definecolor{mygreen}{RGB}{28,172,0} % color values Red, Green, Blue
\definecolor{mylilas}{RGB}{170,55,241}
\usepackage{float}
\usepackage{tabularx}
\usepackage{subfloat}
\usepackage{booktabs}
\usepackage{longtable}
\usepackage{array}
\usepackage{multirow}
\usepackage{wrapfig}
\usepackage{float}
\usepackage{colortbl}
\usepackage{pdflscape}
\usepackage{tabu}
\usepackage{threeparttable}
\usepackage{threeparttablex}
\usepackage[normalem]{ulem}
\usepackage{makecell}
\usepackage{xcolor}
\usepackage{mathtools}

\title{Monetary Policy Rules - Some Notes}
\author{\href{https://hariharanjayashankar.github.io/}{Hariharan Jayashankar}}

\begin{document}

\maketitle

\section{Overview}

This is a document detailing some of the small things I learnt from \citet{clarida2000}. Specifically what I learnt from the model at the end depicting sunspot shocks. This is a very small part of the paper which mainly serves to elucidate why the empirical results might be important. But while going through what should've been a simple model, I found somethings a bit puzzling in the IRFs.

The model is a simple New Keynesian model characterized by 4 equations in equilibrium:
\begin{align}
\pi_t =& \delta \mathbb{E}\{\pi{_t+1} | \Omega_t\} + \lambda (y_t - z_t) \\
y_t =& \mathbb{E}[y_{t+1} | \Omega_t] - \frac{1}{\sigma}(r_t - \mathbb{E}[\pi_{t+1} | \Omega_t]) + g_t \\
r_t^* =& \beta \mathbb{E}[\pi_{t+1} | \Omega_t] + \gamma x_t \\
r_t =& \rho r_t + (1 - \rho) r_t^*
\end{align}

Where $\pi_t$ is the inflation rate at time $t$, $y_t$ is the output at $t$. $z_t$ is the natural rate of output at $t$, $x_t \coloneqq y_t - z_t$. $g_t$ is some exogenous demand factor. We assume $g_t$ and $z_t$ both follow a stationary AR(1) process.


\bibliography{references}
\bibliographystyle{plainnat}


\end{document}
